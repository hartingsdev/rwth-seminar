% This is samplepaper.tex, a sample chapter demonstrating the
% LLNCS macro package for Springer Computer Science proceedings;
% Version 2.20 of 2017/10/04
%
\documentclass[runningheads]{llncs}
% Add your own packages here
\usepackage{graphicx}
%
\begin{document}
%
\title{Hardware Security Tokens With a Focus on FIDO2}
\subtitle{Seminar: Advances in Cryptography and IT-Security}
%
%\titlerunning{Abbreviated paper title}
% If the paper title is too long for the running head, you can set
% an abbreviated paper title here
%

\author{Robert Hartings}

\institute{
\today \\
RWTH Aachen \\
Research Group IT-Security \\[.8cm]
\begin{tabular}{rl}
  \textbf{Organizer:}& Ulrike Meyer\\
  \textbf{Supervisor:}& Vincent Drury\\
\end{tabular}
%\date{01 Apr 2019}
}
%
\maketitle              % typeset the header of the contribution
% The abstract should briefly summarize the contents of the paper in 150--250 words.
\begin{abstract}
The persistent threat situation has led to the need for alternative authentication flows, which are preventing the most common attacks against password-based authentication. FIDO2 was proposed as such an alternative authentication flow by the FIDO Alliance and the World Wide Web Consortium. In FIDO2 the authentication is done via a public-private key pair, which is scoped to the corresponding application, rendering (real-time) phishing and other attacks useless. The private key is saved on an authenticator, which can be roaming or platform bound. USB security key fobs, smartphones and commonly used operating systems can be used as an authenticator. But like all other authentication flows FIDO2 has disadvantages and drawbacks. In this paper these problems are collected from different papers, mainly focusing on possible phishing attacks, weak hardware security tokens, and poor user education. For each problem, a solution or a thought of improvement is presented.

\end{abstract}
%
%
%
\section{Introduction}
Despite their problems with phishing or dictionary attacks, for example, credentials with username and password are still the most common variant of user authentication today. To create secure accounts, it is recommended to use strong passwords that contain upper and lower case letters, numbers, and special characters, and should be at least eight characters long. Also, they should not be reused. The user has to remember these complex passwords or use a password manager, which comes with its own disadvantages. If credentials are reused on multiple accounts, they are vulnerable to credential stuffing attacks, in which an attacker uses stolen username/email and password combination from one service on different services hoping that the victim used the same or similar credentials, making it easier for him to guess the right combination of username/email and password. Username and password are always vulnerable to phishing because it cannot be ruled out that even the most experienced user will make a mistake and enter their credentials on a website owned by an attacker.

This problem is challenged by the FIDO Alliance and the World Wide Web Consortium by providing a possible solution: The Fast Identity Online 2 (FIDO2) standard. The main difference between the proposed standard and the status quo is the paradigm shift from "something a user knows" to "something a user possesses". The FIDO2 standard includes a successor to the Universal 2nd Factor (U2F), which was also developed by the FIDO Alliance, and, in addition to the familiar second factor, also offers the possibility for Single Factor Authentication, therefore making passwords as we know them redundant. The cost, inconvenience, and most users' inexperience with (hardware) security tokens are the current reasons for their low uptake \cite{274547}\cite{9152694}.

Similar to other authentication variants, FIDO2 has its own unsolved problems and drawbacks. In this paper, we will summarize these problems and showcase possible solutions. 

\section{Background}
\subsection{Fast Identity Online 2 (FIDO2)}
The Fast Identity Online 2 (FIDO2) Project is a joint effort by the FIDO Alliance and the World Wide Web Consortium (W3C). It is an open authentication standard succeeding prior work by the FIDO Alliance on Universal 2nd Factor (U2F) \cite{9152694}. It consists of two protocols: The WebAuthn protocol, maintained by the W3C, and the Client to Authentication Protocols (CTAP), which are maintained by the FIDO Alliance. Members of the FIDO Alliance include Amazon, Google, Meta, and Microsoft.

The WebAuthn usage is not limited to the FIDO2 context, because the protocol only specifies a JavaScript-based API used for communication between a WebAuthn relying party application (e.g. websites like google.com, ebay.com, or amazon.com) and a WebAuthn client like a browser (e.g. Chrome, Firefox). The defined API enables the creation of strong, scoped, public-private key pairs, which are used as credentials for user authentication. Through scoping the standard ensures that the key pair can only be accessed by origins belonging to the original relying party. The API is used in two cases. First in case of registration and second in case of authentication. In the case of a registration, a public-private key pair is generated  on behalf of the relying party on the authenticator and is subject to user consent. When the user agrees, the private key is saved on the authenticator and the public key is sent to the relying party, along with additional information like metadata of the used authenticator. In the case of an authentication, the user is presented with a selection menu of accepted credentials and the origin that is requesting these keys. In both cases, the user consent and scoping are enforced by conforming User Agents and the used authenticator \cite{000002}. Nowadays, all major browsers and operating systems support WebAuthn \cite{000001}.

The CTAPs standardize the communication between a client and the external (also known as roaming) authentication devices. The protocols expect the authenticator to obtain evidence of user interaction via a gesture, a pin, a pattern, or biometric data like a fingerprint. Neither is it standardized how the channel between the client and the roaming authenticator is established nor how the transport layer is encrypted. The protocol contains the legacy CTAP1 protocol formerly known as the U2F protocol and the current CTAP2 protocol. The CTAP1 authenticators are also called U2F authenticators and the CTAP2 authenticators are also known as WebAuthn or FIDO2 authenticators. The communication between the client and the roaming authenticator is done using one of the following transport technologies: USB (Universal Serial Bus), NFC (Near Field Communication) or BLE (Bluetooth Low Energy) \cite{000003}\cite{274547}\cite{9099190}.

To sum it up, the main idea of FIDO2 is to use public-private cryptography instead of known credentials like username and password. Furthermore, it creates a public-private key pair unique to a given application or website, which is used to sign challenges from the service and is only generated and stored on the authenticator itself. This is realized through mutual authentication using service identification. In the case of login on a website, the authenticator receives the domain of the requesting website, effectively rendering phishing useless, because a relaying attacker cannot provide the authenticator with the right domain \cite{274610}. Public keys acquired through server breaches can neither be reverted to the original private key nor the secret stored on the authenticator. Furthermore, they cannot be used to determine private keys used for other services.

To ensure quality and security the FIDO Alliance has set up a meta service that can be inquired to verify the authenticator used for the login. The relying party can check if the authenticator meets the FIDO Alliance standards and has known vulnerabilities \cite{9099190}.

\subsection{Hardware Security Tokens (HSTs)}
Hardware security tokens come in different shapes and forms. In the FIDO2 context, they are called authenticators, but to the public, they are also sometimes known as security keys. In FIDO2, a distinction is made between roaming and platform authenticators. In the category of roaming authenticators, to just name a few popular security keys, YubiKeys by Yubico, FIDO Keys by Feitian, and Titan Keys by Google are counted. As required by the CTAP and WebAuthn standard the evidence of user interaction is done via a touch sensor or biometric scanner (most commonly fingerprint sensors). Besides the USB Security fobes, phones and laptops are also roaming authenticators. The two largest hardware security tokens on phones are the Andriod Keystore and Apple TouchID. The communication with the client takes place via USB, NFC, or BLE.
In the category of platform authenticators, Apple TouchID, Andriod Keystore, Windows Hello are counted, but also TPM and Trusted Execution Environment. The main difference between roaming and platform authenticators is that platform authenticators run on the same device as the WebAuthn client and therefore do not use cross-platform communication and CTAP \cite{9152694}. 

To be compliant with the standard the user has to touch a sensor to verify his presence and unlock the secret. The security key fobs are commonly shipped with simple presence sensors, but can also be shipped with biometric sensors, most commonly fingerprint sensors. Nowadays, almost all phones are shipped either with fingerprint sensors or face recognition sensors, or both, and they provide capabilities for pins and patterns.

The hardware security tokens are used to securely store a secret used for cryptographic functions in tamper-resistant storage. The secret never leaves the secure storage and is used for deriving subsequent authentication keys for the creation of public-private key pairs. The derived keys are mainly used to sign challenges received from the relying party application, but can also be used to identify a user \cite{272198}.

\section{Problems of FIDO}
While FIDO2 seems like a good solution, there are currently some disadvantages and problems, that cannot be ignored when evaluating the usability and usefulness of hardware security tokens.

\subsection{Downgrade Attacks}
Besides FIDO different methods of multifactor authentication exist, like One-Time-Passwords (OTP), confirmation SMS and calls, and the usage of recovery codes. Commonly the user can choose between the configured MFA schemes, but except FIDO2 none of the mentioned schemes is secure against real-time phishing. While reviewing Alexa's top 100 websites Ulqinaku et al. found out that most of these websites force users to register at least one different MFA to use FIDO2 in the first place. Effectively creating a vulnerability even when FIDO2 is used because it undermines the security of FIDO2. Only Google with Google's Advanced Protection offers a program not relying on weak MFA, but it is opt-in and not actively advertised on the Google Account Dashboard.

The downgrade attack on FIDO2 was shown by Ulqinaku et al. and requires that the victim has different and weaker MFA registered to his account. The goal of the attacker is to display the user a fake login screen and let the victim choose a weaker MFA scheme, which is vulnerable to real-time phishing. To achieve this goal the attacker has to set up a login page that is comparable to the website from which the attacker wants the login data or sessions. Because the user expects the security key pop-up, the attacker has to display the pop-up on the page as well. The attacker has to know if and when the user has used his hardware security token to continue with his attack, otherwise, a user could get suspicious. Normally the browser would present the pop-up above the webpage containing the domain and ask for the security key, but the WebAuthn standard also contains an API function to detect the presence of an HST without displaying this pop-up. The difference is not easy to spot for a normal user \cite{274610}.

The authors of the paper suggested a solution for this problem to disable weaker alternative schemes, but also mentioned the problem of non-scalable recovery. This recovery adds significant cost to the service provider and would not scale to millions of users, while also providing inconvenience for the user. Registering multiple keys can be another solution but may also be costly and create a barrier. To determine if a recovery is genuine risk-based authentication could be used, but recent studies showed that they can be circumvented. Another solution suggested by the authors was to always show the hint when an application tries to access the HST, making it slightly easier for a possible victim to spot the attack.

\subsection{Threats to HST}
While FIDO2 seems secure it relies on a secure HST. Therefore, if the HST is compromised FIDO2 is not secure anymore. As Pfeffer et al. pointed out, an attacker can get a hold of an access token on the supply chain between a manufacturer and the end-users, either by intercepting the delivery or inserting malicious HST as a manufacturer or a re-seller. An attacker can also buy a genuine token and return malicious HST to the seller on the refund because most sellers will not check if the HST got tempered with. On Malicious HST attack vectors are created with one or more of the following: Firmware Modifications, Hardware Modifications, or Secret Extraction. Through the firmware modification, an attacker can pre-initialize a token or add malicious code/hardware which exploits e.g. USB interfaces. In case of hardware modification, an attacker wires the HST up with a wireless transceiver, like a GSM (Global System for Mobile Communications) or a Bluetooth module. But to hardware modification, token replicas are also counted. These replicas are rather easy to create as instructions are publicly available and can be used without expert knowledge. The main goal of both modifications is to extract or fix secrets, e.g. keys or seeds, of the HST so that it has deterministic derived private keys \cite{272198}. This results in the following attack scenarios.

\paragraph{Run-time secret extraction}
The run-time secret extraction can be subdivided into in-band and out-of-band attacks. In the in-band case, the HST is modified in a way that it leaks secrets through in-protocol (covert) channels like the signature or other channels used in the transaction. In the out-of-band case, the HST sends the secrets via a different covert channel outside of the protocol like Bluetooth or GSM.

\paragraph{Delivery-time secret extraction}
The attacker extracts pre-configured keys or seeds through the hardware of software modifications, which allows him to deterministicly guess the public-private key pairs used in future authentication.

This described attack is only relevant to HST which are shipped with pre-configured secrets by manufacture like YubiKeys. In the case of most HST and also YubiKeys these secrets can be changed by the end-user whenever they want with provided software by the vendor.

\paragraph{Secret fixation}
Another way that allows an attacker to deterministicly guess the generated public-private key pair is by pre-loading the secret, also called fixing it. To permanently fix the secret key hardware or software modifications or both are needed. Otherwise, the attacker can also use the software provided by the vendor to achieve his goal. This attack is also only relevant to keys that are pre-loaded with a secret.

\paragraph{Predictable RNG modifcation}
The Random Number Generator used for the secure derivation of keys can be manipulated to only create predictable by using hardware and software modifications. The RNG can be unintentionally designed weakly by the manufacturer so that no modifications are needed for exploitation. When a weak RNG is used, the attacker can guess the key pair more easily.

\paragraph{Ransom attack}
Like other ransom attacks, this attack appears as a denial of service attack. The HST is manipulated in such a way that it stops operating after some time, demanding a ransom to resume its work or release the stored secrets. This is attack has limited availability for FIDO2, since in most cases recovery codes are generated on FIDO2 key registration. This attack is more feasible for hardware wallets (outside of the scope of this paper).

\paragraph{USB pivoting}
On another note, the HST can not only be used to attack authentication flows but also to attack the whole client via the USB interface. If the HST is equipped with malware it can act as a USB Rubber Ducky (emulate a keyboard) or trigger a buffer overflow.

\paragraph{}
Pfeffer et al. present (already existing) methods to detect a tempered HST.
Modification on hardware or firmware level can be detected with the tamper-evident package using holographic stickers or other methods. But this is only a low level of protection since holographic stickers are easily replaceable and the attacker can be a manufacturer or re-seller. An HST token can be a single-piece cast, like the Yubikeys, or can be opened. The single-piece cast can be easily inspected but can be breakable with household chemicals when not using more chemical-resistant plastic. Openable HST can be inspected by users increasing security by visually comparing manufacturer pictures with the present HST. This process has its downsides as it is error-prone and cumbersome. Signals on the printed circuit board (PCB) are interceptable or manipulatable needing shielding with can be done with a secure CPU or a secure element (external co-processor). In theory, the key never leaves this secure element and therefore cannot be intercepted. To prevent firmware manipulation automatic and manual software verification can be used. In the case of automatic verification, it will be distinguished between local and remote validation. The local validation only validates the integrity by conducting a simple signature check. The remote validation is more sophisticated; the internal status is validated by a third party, making token replicas harder to produce. Both methods need to be visible to the user otherwise the user cannot make related trust decisions.
The manual verification can only be done with some HST and the corresponding software has to be searched as it is not easily findable. Moreover, this method is neither explained nor advertised by vendors on delivery, letting uneducated users in the dark. A way to prevent attacks on pre-configured secrets is to not ship them at all and let the user generate their secrets themselves. But the authors state that manual verification and dispense of pre-configured secrets reduce the user-friendliness of HST, which can result in lower market shares \cite{272198}.

\subsection{Misconceptions}
The study conducted by Lassak et al. \cite{274547} about misconceptions in FIDO2 Biometric WebAuthn shows that users are not yet educated enough to understand the basic functionality of FIDO2 HSTs. There are among others misconceptions about storage location, recovery, and usage of different devices. The study was held online with 42 participants from the UK and US. All of them were older than 18. The participants used their android smartphone as the HST.

\paragraph{Storage Location}
The majority of the participants thought that their biometrics were sent (in an encrypted fashion) to the corresponding service provider. Only 14 participants recognized that the biometrics are stored locally and only 2 figured out that the service provider could not get their biometric data, because he does not have the phone.

Likewise, only 24 participants know/guessed that their biometric data are not affected in the event of a database breach on the services provider site.

Overall, only four participants were confident that their biometric data did not leave the phone when they used it for authentication.

\paragraph{Lost HST}
Because the private key used for authentication is stored on a phone, losing the phone can provide an attacker with access to the accounts if he can unlock the secret with a registered method. 39 out of the 42 participants thought that an attacker needs their biometrics for the authentication while the secret can be unlocked with fallback mechanism likes PIN, pattern, or password, which are easier to obtain for the attacker.

\paragraph{Availability}
If the unlocking of the phone fails via registered biometric data only five participants were aware that they can unlock the HST with a registered fallback option like PIN, pattern, or password. The other participants stated that they do not have set up backup methods at the service provider or they can contact the service provider to recover their account.

\paragraph{Multiple Devices}
Also, there are misconceptions concerning device sharing. One misconception is that is possible to use another device (after registering one's biometric data). Only six participants were aware that the public-private key pairs are tied to the authenticating device and that the biometric is only used to decrypt and unlock the private key used for the authentication process. Transferring the private key from one authenticator to another is not intended in the current WebAuthn/FIDO2 specification. The specification state, that if this kind of behavior is needed a roaming authenticator is needed. On the other hand, at some service providers, it is possible to register more than one HST for a single account, allowing multiple authenticators to grant access to the account.

\paragraph{Delegating Access}
To grant a trusted person access 39 participants answered that it is not possible since the person would not have the required biometric data. The other participants argued that it would be possible to use fallback methods or register the biometric data of the trusted person on the used HST. The latter is right, the token can be unlocked using different biometric data if they are registered on the HST.

\paragraph{Idea of resolving this problem}
The authors stated that services using biometric WebAuthn should inform the user that the biometric data is never sent to, nor stored by, the website, informing them that the biometric is only used on their device. Information about biometric WebAuthn also should point out its convenience, speed, and ease of it. The authors also mentioned that the mental model of password-based sign-in is responsible for many misconceptions and that it needs more research to create notifications changing this model. They make no specific suggestions on how to overcome the discussed problems.

\subsection{More Problems} \label{ref1}
Besides the mention problems the FIDO2 standard has some other problems which are not bound to the implementation. This is highlighted in a study by Lyastani et al. \cite{9152694}. 

\paragraph{Account Recovery}
A fear the participants of the study had, was the loss of the HST, which means that they will not be able to access their accounts. Currently, other 2FA can be used to allow account access even when an HST fails. Some service provides allow for the setup of additional HST, which is recommended by the FIDO Alliance, or other 2FA schemes, which are as we showed earlier vulnerable to e.g. phishing. Only Google's Advanced Protection Program forces the user to register two HSTs.

This issue is not yet challenged by the FIDO Alliance. The current recommendation is to use additional authenticators. The authors recommend guiding the users in the task of scalable account recovery. But this won't change the fact, that account recovery is a serious issue that should be handled by the FIDO instead of a single service provided to create a single way of account recovery. This problem needs attention quickly especially when FIDO2 should become the only 2FA scheme or even a 1FA. 

\paragraph{Account Suspension}
The authors themselves raise the concern of how to revoke the HST from an account if the HST of the user is stolen or lost and the user is not able to access the account using different MFA schemes. The FIDO Alliance argues that the risk of such thefts is lower than being the victim of a phishing campaign or server breach. The authors are concerned about the effect of FIDO (especially OFA) abusers in (intimate) partner violence. They are unsure if HST eases or hampers such target attacks. Also, the authors state that such behavior is not enough considered yet and that users need the possibility to lock their account down without having the HST. They also mentioned that a possible inspiration can be the key revocation in PKI or GPG.

\section{Adoption Barries}
Farke et al. explored the willingness of users to use HST in a study. Since they only had nine participants a general statement cannot be made. But the study can show insights into which reason can prevent a widespreading of HST / FIDO2. The users indicated that they feared losing the HST, the additional effort to plug in the HST and the habit of using passwords.

\paragraph{Convenience}
Three participants stated that the login with an HST requires additional effort and time, making it less convenient than the previously used credentials like username and password. It was also perceived negatively that the setup of the HST took the additional time (5 - 10 minutes). This implies that even a short duration of time can reduce the willingness to use HST.

In the study, the workflow was negatively impacted, because using an HST meant that additional steps were required for each login attempt. The click count and the time to log in were mentioned. Also, it was cited that the current implementation of Windows Hello can be a hurdle for adoption because the account selection when using multiple accounts for a single service was not clear. This behavior is implementation-dependent but if a user has to search the right account in the HST / authentication device it can prevent the spreading of FIDO2. 

The study also indicated that thought must be given to overcoming old habits and that measures must be taken to disseminate HST.

\paragraph{}
The authors mention that the participants understand the authentication process using HST and that the authentication time was more of a limiting factor. The difference in speed could be one of the primary adoption barriers for FIDO2 / HST. Also, participants answered that unconscious decisions to use passwords instead of the HST makes it even more difficult to change the mental mindset of the users, in general, to use HST over passwords. Since only the recent version of operating systems and clients were equipped with all WebAuthn features, WebAuthn has the potential to spread better in the future. Since it is no longer required only special software or certain knowledge and is thus better available to the general public. To overcome the mentioned adoption barriers the authors have the following suggestions:
\begin{itemize}
  \item Support for multiple HST (platform and roaming)
  \item Requirement of adoption of HST
  \item Implementation of FIDO2 in as many as possible services
  \item Allow to "remember" the unlock state of HST when using PIN or biometrics.
\end{itemize}
\cite{255646}

\section{Conclusion}
FIDO2 in combination with hardware security tokens offers a more secure way of user authentication. This authentication scheme is resistant to all currently used attacks on password flows. It even provides the possibility of single-factor authentication, retiring passwords, and their problems. But as we found out, it has its problems and drawbacks. If more websites would use FIDO2 at all or even better as the single 2FA scheme, more users would be forced to use it. The argument of the added cost of authentication through expensive tokens can be invalidated with the fact that today almost everyone already owns a smartphone. In our opinion, simple user education is needed, but this can be done during the setup of the security tokens on the site of the service providers. Also, the problems presented in section \ref{ref1} need to be considered before FIDO2 has a respective market share to avoid problems with recovery and especially to prevent the abuser from accessing an account of a victim.


% Start bibliography
\bibliographystyle{alpha}
\bibliography{literature}

\end{document}
